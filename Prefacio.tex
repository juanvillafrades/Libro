% \titleformat{\chapter}[frame] %para cambiar la forma en la que se muestra el capitulo
% {\normalfont}{\filcenter\small
% \ Capítulo \thechapter \ }
% {7pt}{\Large\bfseries\filcenter}

\chapter*{Prefacio}
\markright{Prefacio}
\addcontentsline{toc}{chapter}{Prefacio}%para que aparezca en la tabla de contenidos
\label{Cap:Prefacio}
Como parte del proceso de divulgación del presente proyecto de grado se realizaron diferentes aportes académicos,  uno de estos fue la participación en la siguiente publicación:
\begin{itemize}
	\item \textbf{MiniMuTe: A muon telescope prototype for studying volcanic structures with cosmic ray flux.} H. Asorey, R. Calderón-Ardila, K. Forero-Gutiérrez, L.A. Nuñez, J. Peña-Rodríguez, J. Salamanca-Coy, D. Sanabria-Gómez, J. Sánchez-Villafrades and D. Sierra-Porta. Scientia  et Technica  Año 2018, Vol. 23, No. 03, septiembre  de 2018. Universidad  Tecnológica  de Pereira. ISSN 0122-1701.
\end{itemize} 
Adicionalmente, parte de los avances y resultados del presente trabajo fueron expuestos en dos eventos nacionales y uno internacional, en (1) como autor y en (2, 3) como coautor.
\begin{enumerate}
	\item \textbf{Control de temperatura para el estudio del voltaje de ruptura de SiPMs}. II Congreso Internacional de Ciencias Básicas e Ingeniería (CICI). Villavicencio, Colombia, Agosto de 2018.
	\item \textbf{MuTe: Un telescopio de muones para estudiar la estructura interna de volcanes mediante el flujo de rayos cósmicos.} V Congreso Colombiano de Astrofísica y Astronomía (COCOA). Pereira, Colombia, Octubre de 2017.
	\item \textbf{Study of the System response, Plastic Scintillator- Optical fiber - SIPM for the MuTe Project.} XIV School on Instrumentation in Elementary Particle Physics (ICFA). European Organization for Nuclear Research (CERN). La Habana, Cuba, Diciembre de 2017.
% 	\item \textbf{MuTe: A hybrid muon telescope for exploring geological structures.} 36th International Cosmic Ray Conference (ICRC). Madison, Wisconsin, USA, Julio de 2019.
\end{enumerate}
\markright{Prefacio}

%%%%%%%%%%%%%%%%%%%%%%%%%%%%%%

