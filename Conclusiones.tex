\chapter{Conclusiones}
\label{Cap:Conclusiones}
En este proyecto se diseñó e implementó un sistema de caracterización de SiPMs compuesto por tres módulos electrónicos que permiten parametrizar el voltaje de ruptura, la ganancia, la tasa de conteo oscuro, el \textit{afterpulses} y el \textit{crosstalk}.\\ \\
El módulo de control de temperatura logró estabilizarse a una temperatura objetivo en el rango de 0 a 50 $^\circ$C en menos de 10 minutos. Asimismo, el módulo de medición de corriente DC permitió realizar mediciones  el rango de  $2.5$ nA a 500 nA y de $2.5$ nA a $1~\mu$A. Por último, la fuente de luz pulzada permitió generar pulsos periódicos de voltaje con una amplitud de entre $1.8$ V y $3.8$ V con un ancho de pulso de entre 6 ns a 11 ns a una frecuencia de entre 100 Hz y $25.6$ kHz y un registro de eventos con una frecuencia de muestreo de 125 MS/s.\\ \\
Por otra parte, en los resultados de la caracterización del SiPM Hamamatsu S13360-1350CS se obtuvo  que la dependencia de la temperatura para el voltaje de ruptura es de $41.7$ $mV/^\circ$C, para la ganancia es de $-1.39\times10^4/^\circ$C y para la tasa de conteo oscuro es de $0.85~kHz /^\circ$C. Se determinó que a 25 $^\circ$C y a un voltaje de polarización de 56 V, la carga de 1 p.e. es $0.206$ pC, el equivalente de 1 p.e. es $13.67$ mV, la probabilidad de \textit{crosstalk} es $5.1$\% y la probabilidad de \textit{afterpulses} es $2.78$\%.\\ \\
Adicionalmente, se encontró que para el SiPM CPTA 151 la dependencia de la temperatura para el voltaje de ruptura es de $66.5$ $mV/^\circ$C, para la ganancia es de $-1.22\times10^4/^\circ$C y para la tasa de conteo oscuro es de $ 4.32~Hz /^\circ$C. Se determinó que a 25 $^\circ$C y a un voltaje de polarización de 53 V la carga de 1 p.e. es $0.175$ pC, el equivalente de 1p.e. es $12.09$ mV, mientras que la probabilidad de \textit{crosstalk} y  \textit{afterpulses} es $19.1$\% y $3.08$\% respectivamente.\\ \\
A modo de comparación entre los SiPMs estudiados, se observó que el Hamamatsu S13360-1350CS presenta una mayor ganancia, una menor dependencia de la temperatura en el voltaje de ruptura y una mayor dependencia de la temperatura en la ganancia. Por otra parte, se observó que en términos de ruido el  Hamamatsu S13360-1350CS presenta un menor nivel de DCR y una menor probabilidad de \textit{crosstalk} y \textit{afterpulses}.\\ \\  
Los resultados de la caracterización del SiPM  S13360 1350-CS de Hamamatsu implican que se debe monitorear la temperatura a la que están operando los SiPMs en MuTe, para a partir del coeficiente de temperatura, realizar la corrección en el voltaje de polarización de los mismos. Así como, establecer un umbral de discriminación mayor a 5 fotones-equivalentes para los eventos que se registran, con el fin de filtrar el ruido asociado a la tasa de conteo oscuro,\textit{crosstalk} y \textit{afterpulses}.\\ \\

% \begin{itemize}
% \item[\textbullet] Conclusión A
% \item[\textbullet] Conclusión B
% \item[\textbullet] Conclusión C
% \item[\textbullet] Conclusión D
% \item[\textbullet] Conclusión E
% \end{itemize}
%%%