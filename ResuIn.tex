\newpage
%\thispagestyle{empty}
%\pagestyle{ruled}
%\thispagestyle{myheadings}
\markright{ABSTRACT}
%\part{RESUMENES}
\begin{center}
\begin{Large}
%\chapter*{RESUMEN DEL TRABAJO PROPUESTO}
\textbf{Abstract}
\end{Large}
\end{center}


\noindent \textbf{Title:} Design and implementation of a characterization system for the silicon photomultipliers (SiPM) of the Muon Telescope (MuTe) \footnotemark[1].\\ 



\noindent \textbf{Author}: \insertautoruno \footnotemark[2].   \\    


 
\noindent \textbf{Key Words:} Silicon photomultipliers, temperature dependence, equivalent photon, PID controller, breakdown voltage, dark count, crosstalk, afterpulse.\\

    
%%%%%%%%%%%%%%%%	DESCRIPCION    %%%%%%%%%%%%%
%\begin{tabular}{ p{15.5cm} }
%\hline
\noindent \textbf{DESCRIPTION}\\
% The MuTe project plans to carry out a telescope that will allow to implement muography of volcanoes in Colombia. Silicon photomultipliers are used in MuTe for detecting photons generated by scintillator bars when charged particles cross them. The implementation of three characterization modules  help to know how temperature and overvoltage affect the SiPM performance. This work shows all the electronics circuits designed in each module as well as the experimental setup for the characterization of breakdown voltage, gain, dark count, crosstalk and afterpulses.\\
% %aside from obtaining of charge and peak histogram and the equivalent photon value.
% The SiPM S13360-1350CS and CPTA 151 were studied in order to validate the developed modules and the characterization protocols. The most outstanding results is the  temperature dependence of the breakdown voltage, the dark count rate, crosstalk and afterpulse probability.\\
% As a conclusion,  working temperature in MuTe should be monitored in order to carry out the operation voltage adjustment usingnthe temperature coefficient. As well as setting a threshold more than 5 equivalent photons for the events registered with the purpose of filter the noise associated with dark count rate, crosstalk and afterpuse.
The MuTe project plans to carry out a telescope that will allow implementing muongraphy of volcanoes in Colombia. Silicon photomultipliers are used in MuTe for detecting photons generated by scintillator bars when charged particles cross them. The implementation of three characterization modules helps to know how temperature and overvoltage affect SiPM performance. This work shows all the electronics circuits designed in each module as well as the experimental setup for the characterization of breakdown voltage, gain, dark count, crosstalk, and afterpulses.\\
The SiPM S13360-1350CS and CPTA 151 were studied to validate the developed modules and characterization protocols. The most outstanding results are the temperature dependence of the breakdown voltage, the dark count rate, crosstalk and afterpulse probability.\\
As a conclusion, the working temperature in MuTe should be monitored to carry out the operation voltage adjustment using the temperature coefficient. As well as setting a threshold more than 5 equivalent photons for the events registered with the purpose of filter the noise associated with dark count rate, crosstalk, and afterpulse.
%\\[0.1cm] %\hline
%\end{tabular}
			
				
\footnotetext[1]{Bachelor Thesis}
\footnotetext[2]{Facultad de Ingenierías Físico-Mecánicas.  Escuela de Ingenierías Eléctrica, Electrónica y  de Telecomunicaciones. Director: \insertdirector, PhD. Codirector: \insertcodirector, PhD.} 
%\footnotetext[3]{Estudiante de Ingeniería Electrónica de la Universidad Industrial de Santander. Código: \insertcodigodos .} 
%\footnotetext[4]{Estudiante de Ingeniería Electrónica de la Universidad Industrial de Santander. Código: \insertcodigouno .} 

%%%%%%%%%%%%%%%%