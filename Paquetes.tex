\usepackage[spanish,es-tabla]{babel}
\usepackage{courier}
\usepackage[T1]{fontenc}
\usepackage{helvet}
\usepackage[utf8]{inputenc} %Para las tildes
\usepackage{geometry}
\geometry{verbose,letterpaper,tmargin=3cm,bmargin=3cm,lmargin=4cm,rmargin=2cm,headheight=25pt,headsep=1cm,footskip=30pt}%márgenes para las normas ICONTEC

\setcounter{secnumdepth}{3} %para que ponga 1.1.1.1 en subsubsecciones
\setcounter{tocdepth}{3} % para que ponga subsubsecciones en el indice
%\usepackage{pslatex}%Usar times new roman
\usepackage{longtable}
\usepackage{pifont}
\usepackage[dvipsnames]{xcolor}
\usepackage{graphicx}%para las imágenes
\usepackage{setspace}

%%%%%%%%%%%%%%%%%%%%%%%%%%%%%%%%	PAQUETE PARA LINKS EN EL PDF Y CPNFIGURACIÓN	%%%%%%%%%%%%%
%\usepackage[pdftex]{hyperref}
\usepackage[linktoc=all]{hyperref}
\hypersetup{bookmarksopen, bookmarksnumbered, colorlinks=true,linkcolor=black, pdftitle=Libro: Diseño e implementación de un sistema de caracterización para los fotomultiplicadores de silicio (SiPM) del telescopio de muones (MuTe),pdfauthor=Juan Carlos Sánchez Villafrades \copyright,pdfsubject=Trabajo de Grado,pdfkeywords=TAI;CTCR;TDS,citecolor=black,urlcolor=blue}

%%%%%%%%%%%%%%%%%%%%%%%%%%%%%		PAQUETES PARA ENCABEZADO Y PIE DE PAG. 	%%%%%%%%%%%%%%
\usepackage{fancyvrb}
\usepackage{fancyhdr}

%\usepackage{lastpage} % Para obtener el valor de la última pagina

\usepackage{amsfonts} %Para símbolos y símbolo números reales

\usepackage{amsmath}
%\numberwithin{equation}{part} %modifica la forma en la que se enumeran las ecuaciones


%%%%%%%%%%%%%%%%%%%%%%%%%%%%%%%%%%%%%%%%%%%%%%%%%%%%%%%%%%%%%%%%%%%%%%%%%%%%%%%%%%%%%%%%%%%%%%%%%%%%%%%%%%%%%%%%%%%%%%%%%%%%%%%%%%%%%%%%%%%%%%%%%%%%%%%%%%%%%%
%\renewcommand{\theequation}{\arabic{equation}} %modifica la forma en la que se enumeran las ecuaciones

% Librería para nombres en listas y configuración de alineación izquierda
% y un solo punto antes de la descripción
% ------------------------------------------------------------------------
\usepackage[labelsep=period]{caption}
\captionsetup{justification=raggedright,singlelinecheck=false}
\renewcommand{\thefigure}{\arabic{figure}} %modifica la forma en la que se enumeran las figuras
% ------------------------------------------------------------------------
% Definición para formato en nombres de figuras
% ------------------------------------------------------------------------
\usepackage{caption}
\usepackage{subcaption}
\usepackage[labelsep=period]{caption}
\captionsetup[figure]{
 labelfont=it
}

%% para numerar en orden usar \setcounter{figure}{N}  , donde N es el número de la imagen anterior

\renewcommand{\thetable}{\arabic{table}} %modifica la forma en la que se enumeran las tablas

% Para alinear el indice de contenidos a la izquierda
\usepackage{tocloft}
% Elimina rellenado con puntos en la tabla de contenidos
% ------------------------------------------------------------------------
\renewcommand{\cftdot}{ }
%lista de anexos
\newcommand{\listappendixname}{\hfill \Large Índice de Anexos \hfill}
\newlistof{appendix}{app}{\listappendixname}
\setcounter{appdepth}{2}    
\renewcommand{\theappendix}{\Alph{appendix}}
\renewcommand{\cftappendixpresnum}{Appendix\space}
\setlength{\cftbeforeappendixskip}{\baselineskip}
\setlength{\cftappendixnumwidth}{1in}

\newlistentry[appendix]{subappendix}{app}{1}
\renewcommand{\thesubappendix}{\theappendix \arabic{subappendix}}
\renewcommand{\cftsubappendixpresnum}{Appendix\space}
\setlength{\cftsubappendixnumwidth}{1in}
\setlength{\cftsubappendixindent}{0em}

\newlistentry[appendix]{subsubappendix}{app}{1}
\renewcommand{\thesubsubappendix}{\theappendix 
\arabic{subappendix}.\arabic{subsubappendix}}
\renewcommand{\cftsubsubappendixpresnum}{Appendix\space}
\setlength{\cftsubsubappendixnumwidth}{1in}
\setlength{\cftsubsubappendixindent}{0em}

\newcommand{\myappendix}[1]{%
  \refstepcounter{appendix}%
  \chapter*{\theappendix\space #1}%
  \addcontentsline{app}{appendix}{ANEXO {\theappendix} \quad #1}%
  \par
}

\newcommand{\subappendix}[1]{%
  \refstepcounter{subappendix}%
  \section*{\thesubappendix\space #1}%
  \addcontentsline{app}{subappendix}{Anexo {\thesubappendix} \quad #1}%
}

\newcommand{\subsubappendix}[1]{%
  \refstepcounter{subsubappendix}%
  \subsection*{\thesubsubappendix\space #1}%
  \addcontentsline{app}{subsubappendix}{Anexo {\thesubsubappendix}\quad #1}%
}

\cftsetindents{chapter}{0em}{1em}
\cftsetindents{section}{0em}{2em}
\cftsetindents{subsection}{0em}{3em}
\cftsetindents{subsubsection}{0em}{4em}

% Para ajustar el titulo de la tabla de contenidos
\renewcommand\cfttoctitlefont{\hfill\Large\bfseries}
\renewcommand\cftaftertoctitle{\hfill\mbox{}}

\renewcommand\cftloftitlefont{\hfill\Large\bfseries}
\renewcommand\cftafterloftitle{\hfill\mbox{}}

\renewcommand\cftlottitlefont{\hfill\Large\bfseries}
\renewcommand\cftafterlottitle{\hfill\mbox{}}

%\renewcommand\cftlatitlefont{\hfill\Large\bfseries}
%\renewcommand\cftafterlistappendixtitle{\hfill\mbox{}}

\setcounter{tocdepth}{4}

%%%%%%%%%%%%%%%%%%%%%%%%%%%%%%%%%%%%%%%%%%%%%%%%%%%%%%%%%%%%%%%%%%%%%%%%%%%%%%%%%%%%%%%%%%%%%%%%%%%%%%%%%%%%%%%%%%%%%%%%%%%%%%%%%%%%%%%%%%%%%%%%%%%%%%%%%
% Elimina indentación o sangría al inicio de un párrafo
% ------------------------------------------------------------------------
\setlength\parindent{0pt}
\onehalfspacing
%\doublespacing

\usepackage{multirow,colortbl}

 \lhead{}
 \rhead{}
\renewcommand{\headrulewidth}{0pt}

% aquí definimos el encabezado y pie de pagina de la pagina inicial de un capitulo.
\fancypagestyle{plain}{
%\fancyhead[L]{K1}
%\fancyhead[C]{K2}
%\fancyhead[R]{K3}
%\fancyfoot[L]{L1}v%  \thepage de \\pageref{LastPage}
\fancyfoot[C]{Página \thepage }
%\fancyfoot[R]{L3}

%\renewcommand{\headrulewidth}{0.5pt}
%\renewcommand{\footrulewidth}{0.5pt}
}

\renewcommand{\footnoterule}{\vspace*{-3pt}
\noindent\rule{16.5cm}{1pt}\vspace*{2.6pt}} %Nota de Pie de pagina (largo y grosor de la linea)


\usepackage{float}% Permite fijar las tablas (y figuras) en un lugar determinado. Con el comando H.


%%%%%%%%%%%%%%%%%%		PARA DAR FORMATO A LOS CAPITULOS	%%%%%%%%%
\usepackage{titlesec}% Permite modificar partes,capítulos,secciones, etc

% \titleformat
% {\chapter} % command
% [display] % shape
% %[frame] % shape
% {\bfseries\Large\itshape \filcenter} % formato al número
% {\chaptertitlename\
% \thechapter} % label
% {0.5ex} % sep
% { %formato a la caja del capitulo
%     \rule{\textwidth}{2pt}
%     \vspace{1ex}
%     \centering %centra la "caja" xD Aja!
% } % before-code
% [
% \vspace{-0.5ex}%
% \rule{\textwidth}{2pt}
% ] % after-code
\usepackage{chngcntr}
\titleformat{\chapter}{\normalfont\normalsize\bfseries\centering}{\thechapter. }{0pt}{}
\titlespacing{\chapter}{0pt}{0pt}{20pt}
% \titleformat
% {\chapter} % command
% [display] % shape
% {\bfseries\Large\itshape} % format
% {\hfill  Capítulo \thechapter \hfill} % label
% {0ex} % sep 
% {
%     %\rule{\textwidth}{0pt}
%     \vspace{0ex}
%     \centering
% } % before-code
% [
% \vspace{-0.5ex}% 
% \rule{\textwidth}{0pt}
% ] % after-code

%Definición de los símbolos para listas itemizadas:
% \AtBeginDocument{
%   \def\labelitemi{\ding{118}}
%   \def\labelitemii{\(\bullet\)}
%   \def\labelitemiii{\ding{223}}
%   \def\labelitemiv{\ding{245}}
% }


\usepackage{multicol} %para usar texto a dos o más columnas

%Para que se habilite la opción de modificar las márgenes en las páginas
\newenvironment{changemargin}[2]{% 
\begin{list}{}{% 
\setlength{\topsep}{0pt}% 
\setlength{\leftmargin}{#1}% 
\setlength{\rightmargin}{#2}% 
\setlength{\listparindent}{\parindent}% 
\setlength{\itemindent}{\parindent}% 
\setlength{\parsep}{\parskip}% 
}% 
\item[]}{\end{list}} 

%% se generan campos para facilitar la edición del archivo
\def\autoruno#1{\gdef\insertautoruno{#1}\gdef\@authoruno{#1}}
\def\autordos#1{\gdef\insertautordos{#1}\gdef\@authordos{#1}}
\def\correouno#1{\gdef\insertcorreouno{#1}}
\def\correodos#1{\gdef\insertcorreodos{#1}}
\def\codigouno#1{\gdef\insertcodigouno{#1}}
\def\codigodos#1{\gdef\insertcodigodos{#1}}
\def\titulo#1{\gdef\inserttitulo{#1}\gdef\@titulo{#1}}
\def\director#1{\gdef\insertdirector{#1}}
\def\cargodirector#1{\gdef\insertcargodirector{#1}}
\def\correodirector#1{\gdef\insertcorreodirector{#1}}
\def\codirector#1{\gdef\insertcodirector{#1}}
\def\cargocodirector#1{\gdef\insertcargocodirector{#1}}
\def\correocodirector#1{\gdef\insertcorreocodirector{#1}}

\usepackage{amsthm}

\theoremstyle{plain}
\newtheorem{prop}{Proposición}[chapter]
\newtheorem{teor}[prop]{Teorema}
\newtheorem{corol}[prop]{Corolario}
\newtheorem{lema}[prop]{Lema}
\theoremstyle{definition}

\newtheorem{defin}{Definición}
%\newtheorem{defin}[num][1]{Definición}
\newtheorem{ejem}{Ejemplo}
\newtheorem{ejer}{Ejercicio}
\theoremstyle{remark}
\newtheorem*{nota}{\textbf{Nota}}
\newtheorem*{notac}{Notación}


\usepackage{amssymb} 

% Librerías que permiten modificar atributos de la tabla de contenido
% ------------------------------------------------------------------------
\usepackage{titletoc}
%\usepackage[subfigure]{tocloft}

% ------------------------------------------------------------------------
% Bibliografía
% ------------------------------------------------------------------------
\usepackage[hyperref=true,
            citereset=none,
            url=false,
            isbn=false,
            backref=true,
            style=verbose-note,
            dashed=true,
            maxcitenames=3,
            maxbibnames=3,
            backend=bibtex,
            block=none,
            defernumbers=true]{biblatex}
\addbibresource{Tesis.bib}         

%\usepackage{cite} % para contraer referencias
%\usepackage[sort&compress]{natbib} % para contraer referencias de texto
%\usepackage{natbib} %anterior
%\usepackage{apacite}
%%%%%%%%%%%%%% Texto en la misma linea %%%%%%%%%%%
\titleformat{\section}{\normalfont\normalsize\bfseries}{\thesection. }{0pt}{}
\titleformat{\subsection}[runin]
  {\normalfont\large\bfseries}{\thesubsection}{1em}{}

\titleformat{\subsubsection}[runin]
  {\normalfont\large\bfseries}{\thesubsubsection}{1em}{}
