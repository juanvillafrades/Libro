\markboth{NOMENCLATURA}{NOMENCLATURA} % encabezado
% \addcontentsline{toc}{chapter}{Nomenclatura}%para que aparezca en la tabla de contenidos
\chapter*{Nomenclatura}

Esta nomenclatura no incluye las notaciones usadas en el apéndice.

Los vectores se denotan mediante caracteres en negrita. \\

\begin{table}[htbp]
\begin{tabular}{l l}
% $A$ & Área de la sección transversal del tubo de Rijke\\
% $\textbf{A},\textbf{B}$ & Matrices de representación en espacio de estados\\
% $c$ & Velocidad del sonido   \\
% $c_p$ & Capacidad Calorífica\\
% $c_v$ & Calor específico a volumen constante\\
% $\textbf{I}$ & Matriz identidad   \\
% $\text{i}$ & Unidad imaginaria   \\
% $K$ & Ganancia de control proporcional\\
% $\textbf{M}$ & Matriz de transferencia para el modelo del tubo de Rijke\\
% $p$ & Presión\\
% $Q$ & Energía térmica \\
% $R_d$ & Coeficiente de reflexión \textsl{downstream}\\
% $R_s$ & Constante específica de los gases   \\
% $R_u$ & Coeficiente de reflexión \textsl{upstream}\\
% $s$ & Variable de Laplace, frecuencia compleja \\
% $\textup{T}$ & Temperatura absoluta   \\
% $T_i$ & Variable de sustitución de Rekasius   \\
% $t$ & tiempo\\
% $u$ & Velocidad del aire \\
% $Z_i$ & Variable de sustitución de Kronecker \\ %$x$ & Coordenada axial del tubo de Rijke   \\
\end{tabular}
\end{table}

\newpage

\textbf{Letras Griegas}

\begin{table}[htbp]
\begin{tabular}{l l}
$\rho$ & Densidad o masa específica \\
$\gamma$ & Relación de la capacidad de calor \\
$\tau$ & Retardo \\
$\Delta$ & Representa cambios o Variaciones \\
$\partial$ & Derivada parcial \\
$\boldsymbol{\wp_0}$ & Núcleo, usado en definición 1 \\
$\boldsymbol{\wp_n}$ & Descendencia, usado en definición 2 \\
$\boldsymbol{\wp}$ & Conjunto Solución, usado en definición 3 \\
\end{tabular}
\end{table}

%\newpage %Necesario porque esta en formato de tablas

\textbf{Superíndices}

\begin{table}[htbp]
\begin{tabular}{l l}
$-$ & Punto de operación, promedio \\
$\sim$ & Variaciones alrededor del punto de operación \\
\end{tabular}
\end{table}

\textbf{Subíndices}

\begin{table}[htbp]
\begin{tabular}{l l}
$ _0$ & Relacionado con el Núcleo (Ver definición 1)\\
$ _n$ & Relacionado con la descendencia (Ver definición 2)\\
$ _1$ & Corte de estudio de la región \textsl{Upstream} \\
$ _2$ &  Corte de estudio de la región \textsl{Downstream} \\
$ _u$ & \textsl{Upstream} \\
$ _d$ & \textsl{Downstream} \\
$ _{mic}$ & Relacionado con micrófono \\
$ _{min}$ & Valor mínimo \\
\end{tabular}
\end{table}